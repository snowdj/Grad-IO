\documentclass[xcolor=pdftex,dvipsnames,table,mathserif,aspectratio=169]{beamer}
\usetheme{metropolis}
%\usetheme{Darmstadt}
%\usepackage{times}
%\usefonttheme{structurebold}

\usepackage[english]{babel}
%\usepackage[table]{xcolor}
\usepackage{pgf,pgfarrows,pgfnodes,pgfautomata,pgfheaps}
\usepackage{amsmath,amssymb,setspace,outline}
\usepackage[latin1]{inputenc}
\usepackage[T1]{fontenc}
\usepackage{relsize}
\DeclareMathSizes{10}{10}{6}{6} 


\title{CCPs and Sufficient Statistics: Hotz-Miller (1993)}
\author{C.Conlon - help from M. Shum and Paul Scott}
\institute{Grad IO }
\date{}
\setbeamerfont{equation}{size=\tiny}

\begin{document}
\begin{frame}
\titlepage
\end{frame}


\section{Before we get started...}

\begin{frame}{From NDP Notes: Policy Iteration (Howard 1960)}
An alternative to value function iteration is policy function iteration. 
\begin{itemize}
\item Make a guess for an initial policy, call it $a(x) = \arg \max_a U(a,x)$ that maps each state into an action
\item Assume the guess is stationary compute the implied $V(a,x)$
\item Improvement Step: improve on policy $a_0$ by solving 
\begin{eqnarray*}
a' = \arg \max_a U(a,x) + \beta \sum_{x'} V(a,x') f (x' | x,a)
\end{eqnarray*}
\item Helpful to define $\tilde{f}(x' | x)$ as transition probability under optimal choice $a(x)$ \alert{post-decision transition rule}.
\item Determine if $\| a' -a\| < \epsilon$. If yes then we have found the optimal policy $a^* $ otherwise we need to go back to step 2.
\end{itemize}
\end{frame}

\begin{frame}{From NDP Notes:  Policy Iteration (Howard 1960)}
\footnotesize
Policy Iteration is even easier if choices AND states are discrete.
\begin{itemize}
\item For Markov transition matrix $\sum_j f_{ij} =1$ ,we want $\pi \, \mathbf{F} = \pi$
\item $\lim_{t \rightarrow \infty} \mathbf{F}^t = \pi$ where the $j$th element of $\pi$ represents the long run probability of state $j$.
\item We want the eigenvalue for which $\lambda = 1$.
\end{itemize}
Now updating the value function is easy for $k$th iterate of PI
\begin{align*}
V^k(x) &= Eu(a^k(x),x) + \beta \, \mathbf{\tilde{F}}^k \, V^k(x)\\
\Rightarrow V^k(x) &= [1 - \beta \, \mathbf{\tilde{F}}^k]^{-1}\, Eu(a^k(x),x)
\end{align*}
\vspace{-.5cm}
\begin{itemize}
\item Very fast when $\beta > 0.95$ and $s$ is relatively small. (Rust says 500 more like 5000).
\item Inverting a large matrix is tricky
\item \alert{This trick is implicit in the HM/AM formulation}.
\end{itemize}
\end{frame}


\section*{Now for real}


\begin{frame}{Motivation}
\begin{itemize}
	\item In Rust, we started with a guess of parameters $\theta$, iterated on the Bellman operator to get $EV_{\theta}(x,j)$ and then constructed CCP's $Pr(a(x) = j | x,\theta) \equiv p(j | x,\theta)$.
	\item A disadvantage of Rust's approach is that it can be computationally intensive
	\begin{itemize}
		\item With a richer state space, solving value function (inner fixed point) 
		can take a very long time,
		which means estimation will take a very, very long time.
	\end{itemize}
	\item Hotz and Miller's idea is to use observable data to form an estimate 
	of (differences in) the value function from conditional choice probabilities (CCP's)
\begin{itemize}
\item We observe $\hat{p}(j | x)$ directly in the data!
\end{itemize}
	\item The central challenge of dynamic estimation is computing continuation values. 
	In Rust, they are computed by solving the dynamic problem.
	With Hotz-Miller (or the CCP approach more broadly), we ``measure'' continuation
	values using a function of CCP's.
	
\end{itemize}
\end{frame}

\begin{frame}
\frametitle{Rust's Theorem 1: Values to CCP's}
\begin{itemize}
	\item In Rust (1987), CCPs can be derived from the value function:
	\begin{align*}
		p_{j}\left( x\right) = \frac{\partial}{\partial \pi_{j} \left(x \right)} W\left( \pi\left( x \right) 
		+\beta E\left[V\left(x' \right)|x,j\right]\right)
	\end{align*}
	where  
	$W\left(u\right) = \int \max_{j} \left\{ u_{j} +\varepsilon_{j}\right\}dG\left( \varepsilon\right)$
	is the surplus function.
	\medskip
	\item For the logit case:
	\begin{align*}
	p_{j}\left(x\right) = \frac{\exp\left(v_{j}\left(x\right)\right)}{\sum_{j'\in J}\exp\left(v_{j'}\left(x\right)\right)}
	\end{align*}
	where the choice specific value function for action $j$ in state $x$ is 
	\begin{align*}
	v_{j}\left(x\right) \equiv \pi_{j}\left(x\right)+\beta E\left[V\left(x' \right)|x,j\right]
	\end{align*}
\end{itemize}
\end{frame}


\begin{frame}
\frametitle{HM's Proposition 1: CCP's to Values}
\begin{itemize}
	\item Notice that CCP's are unchanged by subtracting some constant from every conditional (choice-specific) value function. Thus, consider
	\begin{align*}
	D_{j,0} v\left(x\right) \equiv v_{j}\left(x\right) - v_{0}\left(x\right)
	\end{align*}
	where $0$ denotes some reference action.	
	\medskip
	\item Let $Q:\, \mathbb{R}^{\left| \mathbf{J}\right|-1} \rightarrow \Delta^{\left| \mathbf{J}\right|}$ be the mapping
	from the \alert{differences in conditional (choice-specific) values to CCP's}.
	\medskip
	\item Note: we're taking for granted that the distribution of $\varepsilon$ is identical across states, otherwise
	$Q$ would be different for different $x$.
	\begin{block}{Hotz-Miller Inversion Theorem}
	$Q$ is invertible.
	\end{block}
\end{itemize}
\end{frame}

\begin{frame}
\frametitle{HM inversion with logit errors}
\small
\begin{itemize}
	\item Again, let's consider the case of where $\varepsilon$ is i.i.d type I EV.
	\smallskip
	\item Expression for CCP's: 
	\begin{align*}
	p_{j}\left(x\right) = \frac{\exp\left(v_{j}\left(x\right)\right)}{\sum_{j'\in \mathbf{J}}\exp\left(v_{j}\left(x\right)\right)}.
	\end{align*}
	\item The HM inversion follows by taking logs and differencing across actions: 
	\begin{align*}
		\ln p_{j}\left(x\right) - \ln p_{0}\left(x\right) = v_{j}\left(x\right) - v_{0}\left(x\right)
	\end{align*}
	\item Thus, in the logit case (this looks a lot like Berry (1994)):
	\begin{align*}
	Q_{j}^{-1} \left(\mathbf{p} \right) = \ln p_{j} - \ln p_{0}
	\end{align*}
	\item From now on, I will use $\phi(\mathbf{p})$ to denote $Q^{-1}(\mathbf{p})$.

\end{itemize}
\end{frame}


\begin{frame}{Arcidiacono and Miller's Lemma}
An equivalent result to the HM inversion was introduced by Arcidiacono and Miller (2011). 
It's worth introducing here because it makes everything from now on much simpler
and more elegant.
\begin{block}{Arcidiacono Miller Lemma: Statement}
For any action-state pair $\left(a,x\right)$, there exists a function
$\psi$ such that 
\begin{align*}
V\left(x\right)=v_{a}\left(x\right)+\psi_{a}\left(\mathbf{p}\left(x\right)\right)
\end{align*}
\end{block}
\end{frame}

\begin{frame}{Arcidiacono Miller Lemma: Proof}
\begin{align*}
V\left(x\right) &=  \int\max_{j}\left\{ v_{j}\left(x\right)+\varepsilon_{j}\right\} dG\left(\varepsilon_{j}\right)\\
 &= \int\max_{j}\left\{ v_{j}\left(x\right)-v_{a}\left(x\right)+\varepsilon_{j}\right\} dG\left(\varepsilon_{j}\right)-v_{a}\left(x\right)\\
 &=  \int\max_{j}\left\{ \phi_{ja}\left(\mathbf{p}\left(x\right)\right)+\varepsilon_{j}\right\} dG\left(\varepsilon_{j}\right)-v_{a}\left(x\right)
\end{align*}
Letting $\psi_{a}\left(\mathbf{p}\left(x\right)\right)=\int\max_{j}\left\{ \phi_{ja}\left(\mathbf{p}\left(x\right)\right)+\varepsilon_{j}\right\} dG\left(\varepsilon_{j}\right)$ completes the proof
\end{frame}

\begin{frame}{Important relationships}
\small
\begin{itemize}
\item The Hotz-Miller Inversion allows us to map from CCP's to differences
in conditional (choice-specific) value functions:
\begin{align*}
\phi_{ja}\left(\mathbf{p}\left(x\right)\right)=v_{j}\left(x\right)-v_{a}\left(x\right)
\end{align*}
\item The Arcidiacono and Miller Lemma allows us to relate ex ante and conditional (choice specific)
value functions: 
\begin{align*}
V\left(x\right)=v_{j}\left(x\right)+\psi_{j}\left(\mathbf{p}\left(x\right)\right)
\end{align*}
\item For the logit case:
\begin{align*}
\phi_{ja}\left(\mathbf{p}\left(x\right)\right) & = \ln\left(p_{j}\left(x\right)\right)-\ln\left(p_{a}\left(x\right)\right)\\
\psi_{j}\left(\mathbf{p}\left(x\right)\right) & =  -\ln\left(p_{j}\left(x\right)\right)+\gamma
\end{align*}
where $\gamma$ is Euler's gamma.
\end{itemize}
\end{frame}



\begin{frame}{Estimation example: finite state space I}
\begin{itemize}
\item Let's suppose that $X$ is a finite state space. Furthermore, let's ``normalize'' the
	payoffs for 
	 a reference action $\pi_{0}\left(x\right)=0$ for all $x$. (is this really a ``normalization''?)
	\item Using vector notation (standard font, matrices bold) recall the definition of the choice-specific value function
	for the reference action:
\begin{align*}
v_{0} & = \underbrace{\pi_{0}}_{=0}+ \beta\, \mathbf{F_{0}}\, V = \beta \mathbf{F_{0}}\, V
\end{align*}
\item Using the Arcidiacono-Miller Lemma:
\medskip
\begin{align*}
V-\psi_{0}\left(p\right) & =  \beta\, \mathbf{ F_{0}} \, V\\
\Rightarrow  V & =  \left(\mathbf{I}-\beta\, \mathbf{F_{0}} \right)^{-1} \psi_{0} \left(p\right)
\end{align*}
\end{itemize}
\end{frame}


\begin{frame}
\frametitle{Estimation example: finite state space II}
\begin{itemize}
	\item Now we have an expression for the ex ante value function
	that only depends on objects we can estimate in a first stage:
\begin{align*}
V  =  \left(\mathbf{I}-\beta\, \mathbf{F_{0}}\right)^{-1}\psi_{0}\left(p\right)
\end{align*}
\item To estimate the utility function for the other actions, 	
\begin{align*}
v_{j} & =  \pi_{j}+\beta\, \mathbf{F_{j}} \, V\\
V-\psi_{j}\left(p\right) & =  \pi_{j}+\beta\,  \mathbf{F_{j}} \, V\\
\pi_{j} & =  -\psi_{j}\left(p\right)+\left(\mathbf{I}-\beta\,  \mathbf{F_{j}} \right)\,V\\
\pi_{j} & =  -\psi_{j}\left(p\right)+\left(\mathbf{I}-\beta\,   \mathbf{F_{j}} \right)\, \left(\mathbf{I}-\beta\,  \mathbf{F_{0}} \right)^{-1} \,\psi_{0}\left(p\right)
\end{align*}
\end{itemize}
\end{frame}

\begin{frame}
\frametitle{Identification of Models I}
\begin{itemize}
	\item If we  run through the above argument with $\pi_{0}$ fixed to 
	an arbitrary vector $\widetilde{\pi}_{0}$ rather than 0, we will arrive at the
	following:
	\begin{align*}
		\pi_{j} = \mathbf{A_{j}} \, \widetilde{\pi}_{0} + b_{j} 
	\end{align*}
	where $ A_{j} $ and $b_{j} $ depend only on things we can estimate in a first
	stage: 
	\begin{align*}
	\mathbf{A_{j}} & = \left(1-\beta\, \mathbf{F_{j}}\right) \, \left(1-\beta\,  \mathbf{F_{0}}\right)^{-1}\\
	b_{j} & = \mathbf{A_{j}}\, \psi_{0}\left(p\right)-\psi_{j}\left(p\right)
	\end{align*}
	\item We can plug in any value for $\widetilde{\pi}_{0}$, and each value
	will lead to a different utility function (different values for $\pi_{j}$). Each
	of those utility functions will be perfectly consistent with the CCP's we observe.
\end{itemize}
\end{frame}

\begin{frame}{Identification of Models II}
Another way to see that the utility function is under-identified:
\begin{itemize}
\item If there are $\left|X\right|$ states and $\left|J\right|$ actions,
the utility function has $\left|X\right|\left|J\right|$ parameters.
\item There are only $\left|X\right|\left(\left|J\right|-1\right)$
linearly independent choice probabilities in the data, so we have
to restrict the utility function for identification. 
\item Magnac and Thesmar (2002) make this point as part of their broader
characterization of identification of DDC models:
\begin{itemize}
\item Specify a vector of utilities for the reference action
$\widetilde{\pi}$, a distribution for the idiosycratic shocks $G$,
and a discount factor, and we will be able to find a model rationalizing
the CCPs that features $\left(\widetilde{\pi},\beta,G\right)$.
\end{itemize}
\end{itemize}
\end{frame}

\begin{frame}{Identification of Counterfactuals}
Imposing a restriction like $\forall x:\pi_{0}\left(x\right)=0$
is NOT a normalization: 
\begin{itemize}
\item If we were talking
about a static normalization, each $x$ would represent a different
utility function, and $\pi_{0}\left(x\right)=0$ would simply be a
level normalization. However, in a dynamic model, the payoffs in one
state affect the incentives in other states, so this is a \alert{substantive
restriction.}
\end{itemize}
But do these restrictions affect counterfactuals?
\begin{itemize}
\item It turns out that some (but not all!) counterfactuals ARE identified, in spite of the
under-identification of the utility function. 
\item Whatever value
$\widetilde{\pi}_{0}$ we impose for the reference action, the model will not only
rationalize the observed CCP's but also predict \alert{the same counterfactual CCP's}.
\item Kalouptsidi, Scott, and Souza-Rodrigues (2016) sort out when counterfactuals
of DDC models are identified and when they are not.
\end{itemize}
\end{frame}

\begin{frame}{Extensions}
\begin{itemize}
\item We cheated a bit because we assumed that not only were actions discrete but so was the state space. This trick is often attributed to Pesendorfer and Schmidt-Dengler (ReStud 2008).
\item If the state space is not discrete we need to do some forward-simulation [next slide]. (Hotz, Miller, Sanders, Smith ReStud 1994).
\item Others have extended these ideas to \alert{dynamic games}. See Aguirragabiria and Mira (Ecma 2002/2008) and Bajari Benkard and Levin (Ecma 2007).
\item Srisuma and Linton (2009) [very hard] show how to use Friedholm integral equations of 2nd kind to extend to continuous case.
\end{itemize}
\end{frame}

\begin{frame}{Continuous State Space}
When state space is continuous instead of discrete:\\
Exact Problem
\begin{align*}
V(x) = \max_{a \in A(x)} \left[ (1-\beta)\, u(x,a) + \beta \, \int V(x')\, f(dx' | x,a) \right]
\end{align*}
Approximation to the problem:
\begin{align*}
\hat{V}(x) = \max_{a \in \hat{A}(x)} \left[ (1-\beta) \, u(x,a) + \beta \, \sum_{k=1}^N \hat{V}(x') \, f (x_k' | x,a) \right]
\end{align*}
\begin{itemize}
\item Now we need to do actual numerical integration instead of just summation.
\item We cannot use the $[I- \beta \, \mathbf{F}]^{-1}$ to get the ergodic distribution.
\item Usually requires interpolating between grid points to evaluate $EV(\cdot)$.
\end{itemize}
\end{frame}


\begin{frame}{Forward Simulation}
In practice, "truncate" the infinite sum at some period $T$: 
\begin{align*} 
& \tilde V(x, d=1;\theta) =\\
& u(x, d=1; \theta) + \beta E_{x'|x, d=1} E_{d'|x'} E_{\epsilon''|d', x'} [u(x', d';\theta) + \epsilon' \\
& + \beta E_{x''|x', d''} E_{d''|x''} E_{\epsilon'|d'', x''} [u(x'', d'';\theta) + \epsilon'' + \cdots \\
& \beta E_{x^T|x^{T-1}, d^{T-1}} E_{d^T|x^T} E_{\epsilon^T|d^T, x^T} [u(x^T, d^T;\theta) + \epsilon^T ]]]
\end{align*}
Also, the expectation $E_{\epsilon|d, x}$ denotes the expectation of the $\epsilon$ conditional choice $d$ being taken, and current mileage $x$. For the logit case, there is a closed form:
$$ E[\epsilon | d, x] = \gamma - \log(Pr(d|x))$$
where $\gamma$ is Euler's constant ($0.577\cdots$) and $Pr(d|x)$ is the choice probability of action $d$ at state $x$. \\
\vspace{2mm}
Both of the other expectations in the above expressions are observed directly from the data. 
\end{frame}

\begin{frame}{Forward Simulation}
Choice-specific value functions can be simulated by (for $d=1,2$): 
\begin{align*}
\tilde V (x, d; \theta) \approx & \frac 1 S \sum_s [ u(x, d;\theta) + \beta [ u(x'^s, d'^s;\theta) + \gamma - \log(\hat P(d'^s|x'^s)) \\
& +\beta [ u(x''^s, d''^s;\theta) + \gamma - \log (\hat P (d''^s |x''^s)) + \beta \cdots ]]]
\end{align*}
\vspace{-.4cm}
\begin{itemize}
\item $x'^s \sim \hat G(\cdot | x, d)$ and  $d'^s \sim \hat p(\cdot | x'^s)$ and $x''^s \sim \hat G(\cdot | x'^s, d'^s)$, etc.
\item In short, you simulate $\tilde V (x, d;\theta)$ by drawing $S$ \alert{sequences} of $(d_t, x_t)$ with a initial value of $(d, x)$, and compute the present-discounted utility correspond to each sequence.
\item Then the simulation estimate of $\tilde V (x, d;\theta)$ is obtained as the sample average. 
\end{itemize}
\end{frame}


\begin{frame}{Forward Simulation}
Given an estimate of $\tilde V (\cdot, d; \theta)$, you can get the \alert{predicted choice probabilities}: 
\begin{equation}
\tilde p (d =1 |x ; \theta) \equiv \frac {\text{exp} \left ( \tilde V (x, d=1; \theta) \right )}{\text{exp} \left ( \tilde V (x, d=1 ;\theta) \right ) + \text{exp} \left ( \tilde V (x, d= 0 ;\theta ) \right )}
\end{equation}
and analogously for $\tilde p (d = 0 |x ; \theta)$. 
\begin{itemize}
\item Note that the predicted choice probabilities are different from $\hat p (d|x)$, which are the \alert{actual choice probabilities} computed from the actual data.
\item The predicted choice probabilites depend on the parameters $\theta$, whereas $\hat p (d|x)$ depend solely on the data.
\end{itemize}
\medskip
An obvious estimator minimizes $\arg \min_{\theta} \|  \tilde{p} (d |x ; \theta) -\hat{p} (d |x) \|$
\end{frame}

\begin{frame}{Rust and Hotz-Miller Comparison}
Rust's NFXP Algorithm
\begin{align*}
V_{\theta}(x) &= f(V_{\theta}(x),x,\theta) \Rightarrow  f^{-1}(x,\theta)  \\
P(d | x,\theta) &= g(V_{\theta}(x),x,\theta) \\
P(d | x,\theta) &= g(f^{-1}(x,\theta)) 
\end{align*}
\begin{itemize}
\item At every guess of $\theta$ we solve the fixed point inverse
\item Plug that in to get choice probabilities
\item Evaluate the likelihood
\end{itemize}
\end{frame}

\begin{frame}{Hotz-Miller (1993) to Aguirregabiria and Mira (2002)}
\begin{itemize}
\item Choice probabilities conditional on any value of observed state variables are uniquely determined by the vector of normalized value functions
\item HM show invertibility proposition (under some conditions).
\item If mapping is one-to-one we can also express value function in terms of choice probabilities. 
\begin{eqnarray*}
V_{\theta}(x) &=& h(P(d | x,\theta),x,\theta)\\
P(d | x,\theta) &=& g(V_{\theta}(x),x,\theta) \\
\Rightarrow P(d | x,\theta) &=& g(h(P(d | x, \theta),x ,\theta),x,\theta)
\end{eqnarray*}
\item The above fixed point relation is used in Aguirregabiria and Mira (2002) in their NPL Estimation algorithm.
\end{itemize}
\end{frame}



\begin{frame}{Hotz-Miller (1993) to Aguirregabiria and Mira (2002)}
\begin{eqnarray*}
P^{k+1}(d | x,\theta) &=& g(h(\hat{P^{k}}(d | x, \theta),s ,\theta),s,\theta)
\end{eqnarray*}
\vspace{-0.5cm}
\begin{itemize}
\item Key point here is that the functions $h(\cdot)$ and $g(\cdot)$ are quite easy to compute (compared to the inverse $f^{-1}$).
\item We can substantially improve estimation speed by replacing $P$ with $\hat{P}$ the Hotz-Miller simulated analogue.
\item The idea is to reformulate the problem from \alert{value space} to \alert{probability space}.
\item When initializing the algorithm with consistent nonparametric estimates of CCP, successive iterations return a sequence of estimators of the structural parameters
\item Call this the $K$ stage policy iteration (PI) estimator.
\end{itemize}
\end{frame}

\begin{frame}{Hotz-Miller (1993) to Aguirregabiria and Mira (2002)}
\begin{itemize}
\item This algorithm nests Hotz Miller $(K=1)$ and Rust's NFXP $(K=\infty)$.
\item Asymptotically everything has the same distribution, but finite sample performance may be increasing in $K$ (at least in Monte Carlo).
\item The Nested Pseudo Likelihood (NPL) estimator of AM $(K=2)$ seems to have much of the gains.
\item For games things are more complicated. Pesendorfer and Scmidt-Dengler describe some problems with AM2007.
\item For a modern treatment see Blevins and Dearing (2020).
\end{itemize}
\end{frame}



\end{document}